\chapter{Besluit}
\par We zijn vertrokken uit de template die gebruik maakt van de interframe techniek. Omdat de verwerking ons nog niet helemaal duidelijk was, zijn we daar ook bij gebleven. Het eerste wat we geprobeerd hebben is een ingangssignaal van de camera te tonen op een scherm via het BlackFin bord. Ook DMAs waren nog niet aan ons besteed. Deze zouden pas later hun nut en nood bewijzen. Hier was het grootste struikelblok het frame zelf. Want er zijn tegenstrijdige standaarden voor verschillende toestellen te vinden. Verder zijn er ook delen van de frame data die niet mogen overschreven worden. De meeste tijd in deze fase sloop in het begrijpen van de frame-indeling.

\par Eens we een stabiel beeld verkregen op het scherm was het tijd om een kleur te vervangen door een andere. Iedereen begrijpt dat dit inhoudt dat je kijkt naar elke pixel in het beeld en beslist of deze pixel binnen de grenzen valt van de gekozen kleur, zo ja vervang je ze, zo nee doe je er niks mee. Zoals voorzien werd code geschreven voor de kleurconversie. Toen nog in RGB. En vanaf dat we begonnen met de kleurconversie werd het beeld opnieuw onstabiel. Meer nog, het was afhankelijk van de hoeveelheid blauw die we wilden vervangen. Dus zomaar pixel na pixel verwerken was te traag.

\par Na wat goede raad van een oud docent, Jurgen Baert, zijn we opnieuw aan de slag gegaan, maar via een compleet nieuwe invalshoek. Namelijk, verlies geen tijd in overbodige berekeningen. We deden de kleurdetectie en -conversie zelf in RGB, terwijl het video signaal zelf YCbCr is. Dus verloren we twee keer tijd. En erg veel tijd zelfs, want de conversies tussen kleurenruimtes zijn floating point bewerkingen op een fixed point processor. Erg duur in tijd.

\par Een tweede idee was om meervoudige buffering toe te passen. Daarvoor zijn we dan ook eens erg minutieus de template gaan bestuderen. Initieel waren er 4 frames voorzien om te bufferen. En we hadden op verschillende plaatsen gelezen dat veel bufferen een goed idee was. Daarom werd het aantal buffers met de nodige hindernissen opgetrokken naar 32. Maar helaas niet meteen met een goed resultaat.

\par De betere resultaten kwamen er pas nadat we de schikking van de buffers ook zo wisten te regelen dat er nooit meer een onaangepast frame naar buiten gestuurd werd. Elk frame dat naar buiten ging werd dus verwerkt. Dit had wel een dramatische daling in framerate tot gevolg. Later werd dit echter wat weggewerkt door een betere schikking van de frameselectie voor verwerking. Er werd ook nog geoptimaliseerd op de kleurverwerking, dit was mogelijk omdat Cb geen bovengrens nodig bleek te hebben, aangezien deze grens gewoon de maximale waarde was. Op empirische wijze zijn we dan tot de meest geschikte grenzen gekomen.

\par Verder is er ook een vrij steile leercurve om goed met de BlackFin borden om te leren gaan. Documentatie is moelijk tot onmogelijk te vinden. Maar hoe beter we begrijpen hoe het werkt hoe beter de resultaten zijn. We stellen dus vast dat het echt wel mogelijk is om met de BlackFin borden aan video verwerking en bewerking te doen, zoals te zien is in de code en het resultaat van onze opdracht.