\chapter{Inleiding}

\section{Doelstelling}
\par Het doel van deze opdracht is om via het Matlab pakket kennis te maken met de basisprincipes van videoprocessing.
Deze kennismaking is de basis van de vervolgopdracht waarbij gebruikt gemaakt wordt van een DSPprocessor om live
videobeelden realtime te bewerken. Een videosignaal bestaat uit verschillende frames. Wanneer men dus een videosignaal
wil bewerken dient men dit frame per frame aan te pakken. Elk frame kan gezien worden als een afbeelding waarop \'e\'en of 
meerdere bewerkingen gebeuren. Aan de hand van het Matlab pakket wordt kennis verworven in DSP-technieken om kleuren in
afbeeldingen te detecteren en veranderen. \bigskip

\par In het eerste deel van deze opdracht is het de bedoeling om aan de hand van het programma Matlab inzicht te verwerven
in het bewerken van afbeeldingen (frames). Nadien zal in het tweede deel hetzelfde principe realtime toegepast worden op 
videobeelden aan de hand van een DSP-processor, namelijk de BlackFin BF-561.

\section {Achtergrond}

\subsection{Chromakey}

\par Chromakey, ook wel kleurwaarde genoemd is een techniek die v\'e\'el gebruikt wordt in de tv- en filmwereld. Tijdens het 
filmen wordt als achtergrond een scherm gebruikt dat een bepaalde effen kleur heeft. Vervolgens wordt via een computer of DSP-processor
deze achtergrond veranderd door een andere achtergrond die afkomstig kan zijn van een andere videobron of afbeelding. De meest gebruikte
kleuren voor deze achtergrond zijn blauw en groen. De reden hiervoor is dat de kleuren blauw en groen het meest verschillen in tint (hue) van
de kleur van de menselijke huid. \bigskip

\par De chromakey techniek wordt al gebruikt sinds de jaren '90 om speciale effecten in films mogelijk te maken. Ook bij het 
weerbericht en het journaal maakt men vaak gebruik van chromakey om extra informatie achter de presentator weer te geven. Het is zowel mogelijk
om live te achtergrond te wijzigen als nadien. \bigskip

\par De chromakey techniek brengt ook een aantal problemen met zich mee. Zo speelt de belichting van de set een belangrijke
rol tijdens het filmen. Het vormen van schaduw op het scherm kan ervoor zorgen dat dit van kleur veranderd, namelijk het wordt
donkerder waardoor het door het bewerkingsporgramma niet meer herkend wordt als de kleur die dient gewijzigd te worden. Een ander
probleem dat zich voordoet is dat te dunne objecten (bewegende haren) moeilijker kunnen worden waargenomen door de computer. Dit kan zorgen
voor een ongewenst effect. Tot slot dient er ook op gelet te worden dat de presentator zijn kleding geen element bevat in de kleur van het 
scherm dat gebruikt wordt. Anders zou dit ook vervangen worden, net als de achtergrond.\bigskip

\par Naast het gebruik chromakey in de achtergrond is het ook mogelijk om in een videofragment een bepaalde kleur te vervangen door een andere
kleur. Deze techniek kan toegepast worden om oude zwart-wit films om te vormen naar een minimalistische kleurenfilm. Dit kan gedaan worden door
aan een bepaalde grijswaarde een bepaalde kleur toe te kennen. Hierbij speelt de belichting en schaduwen vaak een nadelige rol.

\subsection {Matlab}

\par Matlab is een computerprogramma dat het mogelijk maakt om allerhande wiskundige toepassingen te simuleren. Aan de basis hiervan ligt de 
programmeertaal M-code. Via Matlab is het mogelijk om afbeeldingen in te lezen en vervolgens hierop bewerkingen uit te voeren. Vervolgens kan 
de bewerkte afbeelding terug worden weergegeven. 