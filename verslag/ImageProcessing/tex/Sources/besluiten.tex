 \chapter{Algemeen besluit}
 
\par Na deze verschillende opdrachten uit te voeren, en zelf ook wat variaties op de oefeningen met Matlab geoefend te hebben is het begrip DSP op vlak van video toch heel wat opgeklaard.
We kunnen reeds een een verzameling aan kleuren binnen vooropgestelde rgb-grenzen vervangen door een andere kleur. Hoe nauwer deze grenzen, hoe minder de afbeelding zal worden be\"invloed. 
Als dit mogelijk is voor een enkel frame, rest ons enkel nog de stap naar real-time processing.
Dit houdt in dat er continue frames binnekomen die dan ook \'e\'en voor \'e\'en ver- en bewerkt moeten worden.
Als we elk zo'n frame als een enkele statische afbeelding kunnen bewerken, kunnen we de reeds opgedane ervaring inzetten om sneller tot ons gewenste resultaat te komen.
Indien de overstap van Matlab naar DSP++ niet te groot is, dan hebben we ons dus reeds voorbereid op de uiteindelijke opgave.